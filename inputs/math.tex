\newcommand\divides{\ensuremath{\mathrel{|}}}

% "Equal by definition" symbol.
\newcommand\eqdef{\mathrel{\overset{\makebox[0pt]{%
  \mbox{\normalfont\tiny\sffamily def}}}{=}}}

\newcommand\funcdef[2]{#1 \mapsto #2}
\newcommand\funcdecl[3]{#1\colon #2 \rightarrow #3}

\newcommand{\integers}{\mathbf{Z}}

\newcommand\st{\;\vert\;}

\newcommand{\totient}{\ensuremath{\phi}}

% If no newline after theorem et al environment, then no indentation either.
% Requires etoolbox.
\makeatletter
\patchcmd{\@endtheorem}{\@endpefalse}{}{}{}
\patchcmd{\endproof}{\@endpefalse}{}{}{}
\makeatother

% Make optional names for theorems et al bold, leaving the rest
% unchanged (assuming amsthm is loaded). Don't know *why* this
% works, but see:
% https://tex.stackexchange.com/questions/43966/how-to-make-the-optional-title-of-a-theorem-bold-with-amsthm
\makeatletter
\def\th@plain{%
  \thm@notefont{}% optional name font same as heading font
  \itshape% body font
}
\makeatother

% Make optional name in remark environment also bold. Cannot just copy first
% line, works but spacing wrong. So just copy full definition; see: 
% https://tex.stackexchange.com/questions/250035/transform-output-theoremstyleremark-from-italics-to-bold
\makeatletter
\def\th@remark{%
  \thm@headfont{\bfseries}%
  \thm@notefont{}% optional name font same as heading font
  \normalfont % body font
  \thm@preskip\topsep \divide\thm@preskip\tw@
  \thm@postskip\thm@preskip
}
\makeatother

\makeatletter
\newtheorem*{rep@theorem}{\rep@title}
\newcommand{\newreptheorem}[2]{%
\newenvironment{rep#1}[1]{%
\def\rep@title{#2 \ref*{##1}}%
\begin{rep@theorem}}%
{\end{rep@theorem}}}
\makeatother

% Math environments (all like theorem).
\newtheorem{theorem}{Theorem}[chapter]
\newtheorem{corollary}[theorem]{Corollary}
\newtheorem{lemma}[theorem]{Lemma}
\newtheorem{definition}[theorem]{Definition} % 'def' cannot be used as environ name.
\theoremstyle{remark}
\newtheorem{remark}[theorem]{Remark}
\newtheorem{example}[theorem]{Example}

\newreptheorem{theorem}{Theorem}
\newreptheorem{lemma}{Lemma}
\newreptheorem{definition}{Definition}

% https://tex.stackexchange.com/questions/16453/denoting-the-end-of-example-remark#comment1002519_32394
\newcommand{\addQEDstyle}[2]{%
  \AtBeginEnvironment{#1}{\pushQED{\qed}\renewcommand{\qedsymbol}{#2}}%
  \AtEndEnvironment{#1}{\popQED}}
\addQEDstyle{remark}{$\triangle$}
\addQEDstyle{example}{$\lozenge$}

% Makes emphasis in definitions et al. yield upright bold text. Requires etoolbox.
% remark is exempted because its default text is not italic -- so ok
% to use italics for emphasis.
\AtBeginEnvironment{theorem}{\renewcommand\em{\upshape\bfseries}}
\AtBeginEnvironment{definition}{\renewcommand\em{\upshape\bfseries}}
\AtBeginEnvironment{lemma}{\renewcommand\em{\upshape\bfseries}}
\AtBeginEnvironment{corollary}{\renewcommand\em{\upshape\bfseries}}
% \AtBeginEnvironment{remark}{\renewcommand\em{\upshape\bfseries}}

% To get a black filled square at \end{proof}.
\renewcommand\qedsymbol{$\blacksquare$}

\newcommand\swapifbranches[3]{#1{#3}{#2}}
\makeatletter
\MHInternalSyntaxOn
\patchcmd{\DeclarePairedDelimiter}{\@ifstar}{\swapifbranches\@ifstar}{}{}
\MHInternalSyntaxOff
\makeatother
% https://tex.stackexchange.com/questions/94410/easily-change-behavior-of-declarepaireddelimiter

% I can't believe these two are not default...:
\DeclarePairedDelimiter\ceil{\lceil}{\rceil}
\DeclarePairedDelimiter\floor{\lfloor}{\rfloor}

\DeclarePairedDelimiter\parens{\lparen}{\rparen}
\DeclarePairedDelimiter\sparens{\[}{\]}
\DeclarePairedDelimiter\cparens{\{}{\}}
\DeclarePairedDelimiter\abs{\lvert}{\rvert}
\DeclarePairedDelimiter\norm{\lVert}{\rVert}

